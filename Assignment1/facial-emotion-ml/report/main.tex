
% ----------------------------------------------------------
% University of Central Lancashire - Final Year Project Proposal
% Professional LaTeX Template
% ----------------------------------------------------------
\documentclass[12pt,a4paper]{article}

% --- Packages ---
\usepackage[utf8]{inputenc}
\usepackage[T1]{fontenc}
\usepackage[english]{babel}
\usepackage[a4paper,margin=2.5cm]{geometry}
\usepackage{setspace}
\usepackage{graphicx}
\usepackage{titlesec}
\usepackage{fancyhdr}
\usepackage{hyperref}
\usepackage{xcolor}
\usepackage{csquotes}
\usepackage{biblatex}

% --- Setup bibliography ---
\addbibresource{references.bib}

% --- Page Style ---
\pagestyle{fancy}
\fancyhf{}
\rhead{University of Central Lancashire}
\lhead{Final Year Project Proposal}
\cfoot{\thepage}

% --- Title Formatting ---
\titleformat{\section}
{\normalfont\Large\bfseries}
{\thesection}{1em}{}

\titleformat{\subsection}
{\normalfont\bfseries}
{\thesubsection}{1em}{}

% --- Line Spacing ---
\onehalfspacing

% --- Hyperlink setup ---
\hypersetup{
	colorlinks=true,
	linkcolor=black,
	urlcolor=black,
	citecolor=black
}

% --- Title Page ---
\begin{document}
	
	\begin{titlepage}
		\centering
		\vspace*{2cm}
		% --- University Logo Placeholder ---
		\includegraphics[width=8cm]{uclan-logo-placeholder.png}\\[1cm]
		
		{\Huge \bfseries Artificial Intelligence}\\
		{\large Assessment 1}\\[0.25cm]
		\vspace*{2cm}
		{\Huge \bfseries Facial Emotion Recognition using ML}\\
		
		\large \textbf{Sarah Gosch}\\[2cm]
		
		{\large BSc (Hons) Software Engineering}\\[0.25cm]
		{\large School of Engineering and Computing}\\[0.25cm]
		{\large University of Central Lancashire}\\[1.5cm]
		
		{\large \today}
		
		\vfill
	\end{titlepage}
	
	% --- Table of Contents ---
	\tableofcontents
	\newpage
	
	% --- Sections ---
\section{Introduction}
Facial emotion recognition (FER) is the process of detecting and classifying emotions from human facial expressions. 
It is a crucial aspect of human–computer interaction, enabling systems to interpret non-verbal cues. 
This project implements a traditional machine learning approach using Support Vector Machines (SVM), 
K-Nearest Neighbours (KNN), and Decision Trees to classify basic emotions such as \textit{Neutral, Angry, Happy, Surprise, Sad, and Fear}. 
Two datasets, JAFFE and Cohn–Kanade (CK+), are used for evaluation.

\section{Literature Review}


\section{Datasets}
\subsection{JAFFE Dataset}
Contains 213 grayscale images from 10 female subjects expressing 7 emotions. 
Resolution: 256×256 pixels. 

\subsection{CK+ Dataset}
The CK+ dataset includes 593 sequences from 123 subjects. 
The last frame of each sequence is used for classification.

\subsection{Preprocessing}
Faces were detected and cropped using OpenCV’s Haar Cascade. 
All images were resized to 100×100 pixels and normalized. 
To address class imbalance, SMOTE oversampling was applied.

\section{Model Development}
\subsection{Feature Extraction}
Two hand-crafted methods were used:
\begin{itemize}
	\item \textbf{HOG:} Captures edge orientation patterns.
	\item \textbf{LBP:} Encodes local binary texture features.
\end{itemize}

Dimensionality reduction using PCA retained 95\% variance.

\subsection{Classifiers}
Four models were trained:
\begin{itemize}
	\item SVM (RBF kernel)
	\item KNN (k=5)
	\item Decision Tree
	\item Naive Bayes
\end{itemize}

A 5-fold cross-validation approach was used for hyperparameter tuning.

\section{Model Evaluation}
\subsection{Metrics}
Models were evaluated using accuracy, F1-score, and confusion matrices.

\subsection{Results}


\subsection{Discussion}
SVM achieved the best performance overall. 
LBP features were more effective for JAFFE, while HOG performed better on CK+ due to varied lighting and subject diversity. 
Most confusion occurred between \textit{Neutral} and \textit{Sad} classes.

\section{Conclusion}
The experiment demonstrated that classical ML techniques can effectively classify emotions in facial images. 
SVM with HOG or LBP features achieved accuracies above 85\%. 
Future work may integrate temporal information from videos and explore hybrid (shallow+deep) models.

\section*{Supplementary Material}
Source code and data preprocessing scripts:  

\url{https://github.com/yourusername/facial-emotion-ml}

	
\end{document}
