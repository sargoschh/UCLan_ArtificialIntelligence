% Main Content Chapters

%========================================
% CHAPTER 1: INTRODUCTION
%========================================
\chapter{Introduction}

Introduction (10\%): Describe the problem you are working on, and an overview of your results



%========================================
% CHAPTER 2: STATE OF THE ART
%========================================
\chapter{Literature review}

 
Literature review (10\%): Discuss published work that relates to this project. How is your approach similar or different from others?


%========================================
% CHAPTER 3: METHODOLOGY
%========================================
\chapter{Datasets}

Datasets (10\%): Describe the data you are working with for your project. What type of data is it? Where did it come from? How much data are you working with? Did you have to do any preprocessing, filtering, or other special treatment to use this data in your project? If you are collecting new data, how will you do it and incorporate it into your model?

\section{Another Example Section}

\hl{As an example of a figure, consider Figure~}\ref{fig:technical}.

\hl{Each figure is numbered automatically, and it is good academic writing to cite the figure in the body of your text (e.g. "see Figure~}\ref{fig:technical}\hl{", or "Figure~}\ref{fig:technical}\hl{ illustrates a highly technical diagram").}

\begin{figure}[h]
    \centering
    \includegraphics[width=0.8\textwidth]{figures/highly-technical-diagram.png}
    \caption{Highly Technical Diagram}
    \label{fig:technical}
\end{figure}

%========================================
% CHAPTER 4: DESIGN
%========================================
\chapter{Model Development}

 
Model Development (30\%): What method or algorithm are you proposing/using? 
Discuss your approach for solving the given problems. Why is your approach the right thing to do? Did you consider alternative approaches? If you are using any existing implementations, how will you use them and justify the reason for using it to solve the given problem? How have you planned to improve or modify such implementations? 
Explain the inner workings of your developed/used model and justify why and how the model works to solve the given problem. You should demonstrate that you have applied ideas and skills built up during the semester to tackling the given problem. It may be helpful to include figures, diagrams, or tables to describe your method.

%========================================
% CHAPTER 5: IMPLEMENTATION
%========================================
\chapter{Model Evaluation}

 
Model Evaluation (30\%): How will you evaluate your results? Discuss the results obtained from your model. Qualitatively, what kind of results do you expect (e.g., plots or figures)? Quantitatively, what kind of analysis will you use to evaluate and/or compare your results (e.g., what performance metrics or statistical tests)? Discuss the experiments that you performed to demonstrate that your approach solves the problem. You might compare with previously published methods, perform an ablation study to determine the impact of various components of your system, experiment with different hyperparameters or architectural choices, use visualization techniques to gain insight into how your model works, discuss common failure modes of your model, etc. You should include graphs, tables, or other figures to illustrate your experimental results.

\begin{lstlisting}[language=C++, caption=Example Code]
using System;

namespace HiWorld
{
    internal static class Program
    {
        private static void Main(string[] args)
        {
            Console.WriteLine("Hello World!");
        }
    }
}
\end{lstlisting}

\begin{longtable}[!h]{c||c|c|c}  %[!h] steht für here, damit Tabelle an dieser Stelle eingefügt wird, ! für erzwungen
	\hline
	\textbf{Test ID} & \textbf{Description} & \textbf{Expected} & \textbf{Result} \\ %\textbf macht die Elemente fettgeschrieben
	\hline
	\hline
	 &  &  &  \\
	\hline
	 &  &  &  \\
	\hline
	 &  &  &  \\
	\hline
	 &  &  &  \\
	\hline
	\caption{Testing Summary}
	\label{tbl:testing}
\end{longtable} 

\hl{Example for List:}

\begin{itemize}
	\item Point 1
	\item Point 2
	\item Point 3
\end{itemize}



%========================================
% CHAPTER 6: EVALUATION, DISCUSSION AND CONCLUSIONS
%========================================
\chapter{Conclusions}

 
Conclusion (10\%): Summarize your key results - what have you learned? Suggest ideas for future extensions or new applications of your ideas.

